\input{../common/header.tex}
\subtitle{Start programming in Python}
\date{2020-04-02}
\input{../common/title.tex}

\begin{frame}
	\tableofcontents
\end{frame}

\section{Intro}

% variables
% functions
% if, for (while)

\section{Leftover from last week}

\begin{frame}[fragile]{Using Jupyter Notebooks}
	
	If you want to use conda environments in your notebooks, do the following:
	\begin{itemize}
		\item Install \textit{ipykernel} in your environment
		
		\begin{verbatim}
		conda activate <conda-environment>
		conda install ipykernel
		conda deactivate
		\end{verbatim}
		\item Additionally, install \textit{nb\_conda\_kernels} in \textit{base} if you run your jupyter notebook from base:
		\begin{verbatim}
		conda activate base    # could be also some other environment
		conda install nb_conda_kernels
		\end{verbatim}
	\end{itemize}
\end{frame}

\begin{frame}[fragile]{A hint on replacements in exercises and <>}
	
	If you see text enclosed in <> on our slides or in the notebook such as
	\begin{verbatim}
	<something here>
	\end{verbatim}
	it means that 'something here' should be replaced by some input from your side. (Hopefully obvious from the context)
	
	Anything else can just be copied.

\end{frame}


\section{Homework assignment}


\begin{frame}[fragile]{Homework assignment}

    Write a function, which calculates the number of new infections at time $t$.

    \[
        \frac{e^{-(x-\mu)/s}} {s\left(1+e^{-(x-\mu)/s}\right)^2}
    \]

    Then use the following code to plot it:
    % TODO would be nice to prepare a jupyter notebook with widgets to adjust parameters!
\end{frame}

\end{document}
