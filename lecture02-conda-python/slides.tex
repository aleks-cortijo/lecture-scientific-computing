\input{../common/header.tex}
\subtitle{Installation of Python using Conda and first steps in Python}
\date{2020-03-26}
\input{../common/title.tex}

\begin{frame}
	\tableofcontents
\end{frame}


\section{Organizational matters}
\begin{frame}[fragile]{Organizational matters}

	\begin{itemize}
		\item Please participate in writing today's lecture notes:
            \href{https://yourpart.eu/p/lecture-scientific-computing02-notes}{https://yourpart.eu/p/lecture-scientific-computing02-notes}\pause
        \item Let's also try to write a glossary of new terms:\\
            \href{https://yourpart.eu/p/lecture-scientific-computing-glossary}{https://yourpart.eu/p/lecture-scientific-computing-glossary}
		\item Testing and Grading
			\begin{itemize}
				\item Do your homework in your group! We'll check repositories. (Relevant for grading)
				\item Presentation at least once in the online lecture - either a homework exercise or a lecture exercise.
				\item Review tests at the start of each class (starting in coming week). Relevant for you, not for grading.
				
			\end{itemize}
	\end{itemize}
\end{frame}

\section{Conda}
\begin{frame}{Download...}

	Please download Anaconda from\\
	\href{https://www.anaconda.com/distribution/}{https://www.anaconda.com/distribution/}\\

\end{frame}

\begin{frame}[fragile]{Package manager and Environments}
	\begin{itemize}
		\item A package (library or application) is a collection of code that helps you accomplish
            tasks (a bit similar to the \verb|setup.exe| in Windows).
		\item There are e.g. packages for machine learning, for plotting data, for working with tabular data, or for working with matrix-data. More on that later!
		\item A package manager is a convenient way to install software: it resolves all
            dependencies (=other packages needed by the package you want to install), downloads all packages and installs them.
		\item Conda also allows to create environments: within an environment, you are free to install a different Python version, different packages and package versions etc. This helps in having a clean separation between projects. (But also means that you may have installed Python many times on your computer)
	\end{itemize}

\end{frame}


\begin{frame}{Package Manager of our choice}

	\begin{itemize}
		\item We are going to use conda as package manager.
		\item Anaconda is a system that packages a lot of software together with conda and runs on Windows and Linux. It also has a graphical user interface. In class, we use the command line only. Anaconda provides a lot of software, also for R, you may explore it by starting the graphical user interface. (we do not dig deeper here).
		\item Miniconda is an alternative to Anaconda. It is smaller and just comes with the core libraries which are necessary to use conda.

	\end{itemize}


\end{frame}

\begin{frame}[fragile]{Anaconda installation}

		\begin{itemize}
		\item Run setup program
		\item When asked check the checkbox to set the PATH variable
		\item Open git bash and type

		\begin{verbatim}
			conda init bash
		\end{verbatim}
		\item If this works, you successfully installed conda!

		\end{itemize}

\end{frame}

\begin{frame}[fragile]{Using conda environments}

	\begin{itemize}
		\item A conda environment allows you to install a separate version of Python (or any other conda supported software) with associated packages

		\item List all available environments
		\begin{verbatim}
		conda env list
		\end{verbatim}

		\item There should be a $base$ environment available on all installations. It is automatically activated when you use conda!
		\item We now first use conda to $update$ - conda!
		\begin{verbatim}
		conda update conda
		\end{verbatim}
		\item This will download and install the newest version of conda

	\end{itemize}

\end{frame}

\begin{frame}[fragile]{Creating and activating environments}

	\begin{itemize}
		\item You can create a new Python environment with this command
		\begin{verbatim}
		conda create -n <environment-name> python=3.7 anaconda
		\end{verbatim}

		\item \textit{<environment-name>} can be chosen by you. We use \textit{scientific-computing}:
		\begin{verbatim}
			conda create -n scientific-computing python=3.7 anaconda

		\end{verbatim}

		\item Again list all available environments
			\begin{verbatim}
			conda env list
			\end{verbatim}
		\item There should be an environment \textit{scientific-computing} available now.
		\item How to work with it?
		\begin{verbatim}
		conda activate scientific-computing
		\end{verbatim}
		\item Observe how the command line indicates the active environment:
            \textit{(scientific-computing)} is now displayed
		\item Not activating the correct environment is a very common source of problems! Check your environment!

	\end{itemize}

\end{frame}

% anaconda vs miniconda
% what is a package manager?
% environments and requirements in the env.yml files

\section{Jupyter}

\begin{frame}{What is a Juypter Notebook?}

	\begin{itemize}
		\item In principle, you can write your code in any test editor and then run it on the command line
		\item A much better option would be to use an integrated development environment (IDE) such
            as PyCharm (syntax highlighting, code linting, etc.).
		\item A different form of coding environment are so called \textit{Jupyter Notebooks}
		\item Here, the code is written in the web browser, sent to a server, executed there, and results are displayed in the notebook again.

	\end{itemize}


\end{frame}

\begin{frame}{Pros \& Cons of Jupyter Notebooks}

\begin{itemize}
		\item Advantages
	\begin{itemize}
		\item Integration of code and visualization of results: for data analysis crucial.
		\item Interactive environment for data assessment: data (variables, objects, libraries) stay in memory and code can be dynamically adapted (Similar to R)
		\item Code an be run on server with large computational capacity (if available)

	\end{itemize}
	\item Disadvantages
	\begin{itemize}
		\item Managing large junks of code (such as functions) becomes complicated.
		\item Using Jupyter notebooks with version control is difficult, as they introduce their own XML-syntax, which is not really human readable.
		\item Interactive session: very error-prone, as objects and variables can change their state depending e.g. on the number of times a certain cell is executed.
	\end{itemize}

	\item Best of two worlds: use an IDE to manage your stable code (functions etc.) and use notebooks for exploration.
	\item For teaching purposes, notebooks are VERY useful. We are going to use them therefore - and may, at some point, introduce additional software such as an IDE.
\end{itemize}

\end{frame}

\begin{frame}[fragile]{Using Jupyter Notebooks}
		In git bash, type
		\begin{verbatim}
		jupyter notebook
		\end{verbatim}
		This will start the Jupyter Notebook server within the current working directory as a background process and starts a browser.

		To stop it type
		\begin{verbatim}
		CTRL-C
		\end{verbatim}

		If you want to use conda environments in your notebooks, do the following:
		\begin{itemize}
			\item Install \textit{ipykernel} in your environment

		\begin{verbatim}
		conda activate <conda-environment>
		conda install ipykernel
		conda deactivate
		\end{verbatim}
		\item Additionally, install \textit{nb\_conda\_kernels} in \textit{base} if you run your jupyter notebook from base:
		\begin{verbatim}
conda activate base    # could be also some other environment
conda install nb_conda_kernels
		 \end{verbatim}
		\end{itemize}
\end{frame}

\begin{frame}[fragile]{Standard Browser and Jupyter Notebooks}

	jupyter notebook may start with an unwanted Browser. Two things you should try:
	\begin{itemize}
		\item Set your desired Browser as Standard Browser in Windows Settings (Control Panel - Apps - Default Browser)
		\item try
			\begin{verbatim}
		git config --global web.browser google-chrome	\end{verbatim}

	\end{itemize}

\end{frame}

\begin{frame}{Some interesting links}
	\begin{itemize}
		\item \href{https://www.edureka.co/blog/wp-content/uploads/2018/10/Jupyter_Notebook_CheatSheet_Edureka.pdf}{Jupyter notebook cheatsheet}

		\item \href{https://www.preisjaeger.at/deals/gratis-it-workshops-e-books-von-packtpubcom-227041}{Free Python class}

		\item \href{https://docs.python-guide.org/}{The Hitchhiker's Guide to Python}

	\end{itemize}


\end{frame}

% how to use jupyter as a calculator
% the dangers of using jupyter

\section{Homework assignment}

\begin{frame}[fragile]{Homework assignment (I)}

	\begin{itemize}
		\item During your last homework assignment, one of your team members forked the homework repository:\\
            https://github.com/inwe-boku/homework-scientific-computing
		\item Fetch the latest changes from the upstream:
            {\scriptsize
            \begin{verbatim}
cd path/to/homework-scientific-computing
git remote add upstream https://github.com/inwe-boku/homework-scientific-computing/
git pull --no-edit upstream master\end{verbatim}
            }
		\item Install conda and Jupyter on your computer, run Jupyter in the directory of the
            forked repository.
		\item Create a new Notebook and use Python to create an encrypted file with your full name, the registration number (Matrikelnummer), and the git username. For that purpose, use the public key \textit{public\_key.pem} stored in the homework repository in \textit{homework02-conda-python}
		\item The filename of the encrypted file should be
            \textit{group-member-<github-account-name>}, where \textit{<github-account-name>} is
            your Github name.
		\item Use the Python code on the next slides to accomplish this task.
		\item Add the encrypted file to your repository, commit the change, and push it to your
            fork (no pull request).
	\end{itemize}


\end{frame}

\begin{frame}[fragile]{Homework assignment (II)}

	\begin{verbatim}
"""Store personal data encrypted with our public key."""

import subprocess

# if you get an ImportError, you missed to install
# the package cryptography with conda
from cryptography.hazmat.backends import default_backend
from cryptography.hazmat.primitives import serialization
from cryptography.hazmat.primitives import hashes
from cryptography.hazmat.primitives.asymmetric import padding

# change this:
real_name = "First Last"
github_account = "mynickname"
regstration_number = "123456"\end{verbatim}

\end{frame}

\begin{frame}[fragile]{Homework assignment (III)}

	\begin{verbatim}
# open and read the public key
with open("public_key.pem", "rb") as key_file:
    public_key = serialization.load_pem_public_key(
        key_file.read(), backend=default_backend())

def git_config_value(param):
    raw_value = subprocess.check_output(
        ['git', 'config', f'user.{param}'])
    return raw_value.decode().strip()

# determine git name/mail address
git_name = git_config_value('name')
git_mail = git_config_value('email')

# concatenate strings as CSV format
name_mapping = ";".join((real_name, github_account,
                         regstration_number, git_name, git_mail))\end{verbatim}

\end{frame}

\begin{frame}[fragile]{Homework assignment (IV)}
	\begin{verbatim}
# encrypt string
encrypted = public_key.encrypt(
    name_mapping.encode(),
    padding.OAEP(
    mgf=padding.MGF1(algorithm=hashes.SHA256()),
    algorithm=hashes.SHA256(),
    label=None))

# write encrypted data to disk
out_filename = f"group-member-{github_account}.txt"
with open(out_filename, 'wb') as f:
    f.write(encrypted)\end{verbatim}
	\end{frame}

\end{document}
